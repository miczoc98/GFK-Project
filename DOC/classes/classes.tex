\documentclass{article}

\usepackage{amsmath, amssymb}
\usepackage{float}
\usepackage{graphicx}
\usepackage{subcaption}
\usepackage{color}
\usepackage[table,xcdraw]{xcolor}

\usepackage{polski}

\usepackage{listings}


\definecolor{dkgreen}{rgb}{0,0.6,0}
\definecolor{gray}{rgb}{0.5,0.5,0.5}
\definecolor{mauve}{rgb}{0.58,0,0.82}

\lstset{frame=tb,
  language=C++,
  aboveskip=3mm,
  belowskip=3mm,
  showstringspaces=false,
  columns=flexible,
  basicstyle={\small\ttfamily},
  numberstyle=\color{gray},
  keywordstyle=\color{blue},
  commentstyle=\color{dkgreen},
  stringstyle=\color{mauve},
  breaklines=true,
  breakatwhitespace=true,
  tabsize=3,
}
%

\addtolength{\topmargin}{-.875in}
\addtolength{\textheight}{1.75in}


\begin{document}
	\section{Klasy}
		\subsection{LissajousCurve}
			Klasa reprezentująca krzywą Lissajous zapisaną w postaci:
			$$
				x = Asin(nt + \Phi)
			$$
			$$
				y = Bsin(mt + \Psi)
			$$
			$$
				z = Csin(kt + \Theta)
			$$
			lub
			$$
				r = Asin(nt + \Phi)
			$$
			$$
				\phi = Bsin(mt + \Psi)
			$$
			$$
				\theta = Csin(kt + \Theta)
			$$
			\subsubsection{Metody}
				\begin{itemize}
					\item Metoda \lstinline|Point get_pos(double t, bool as_cartesian)| zwraca położenie punktu dla wybranej wartości $t$, we współrzędnych kartezjańskich (\lstinline|as_cartesian = true|) lub sferycznych
							(\lstinline|as_cartesian = false|)	.
					\item Settery pozwalają ustawić parametry krzywej.
				\end{itemize}
			\subsection{Pola}
				Klasa zawiera pola odpowiadające parametrom krzywej tj. $A, B, C, n, m, k, \Phi, \Psi, \Theta$.	
		\subsection{CurveGenerator}
			Klasa odpowiedzialna za generowanie krzywej w postaci kolejki dwukierunkowej złożonej z odcinków.
			\subsubsection{Metody}
				\begin{itemize}
					\item \lstinline|std::deque<Segment> get_next()| zwraca kolejkę zawierającą krzywą.
					\item \lstinline|void set_cartesian(bool b)| pozwala ustawić zmienną \lstinline|m_is_cartesian| określającą czy krzywa jest we współrzędnych kartezjańskich czy sferycznych.
					\item \lstinline|void set_animate(bool b)| ustawia zmienną \lstinline|m_animate| określającą czy krzywa będzie animowana.
					\item \lstinline|void reset()| usuwa całą wygenerowaną krzywą (wywoływana np. po zmianie parametrów).
					\item \lstinline|Segment generate_segment()| pozwala wygenerować kolejny odcinek krzywej.					
					\item Settery parametrów krzywej pozwalają na ustalenie jej kształtu.
				\end{itemize}
			\subsubsection{Pola}
				\begin{itemize}
					\item \lstinline|double m_t| to obecna wartość parametru $t$ ze wzoru krzywej.
					\item \lstinline|int m_current_segment_count| określa liczbę wygenerowanych odcinków.
					\item \lstinline|int m_max_segment_count| to maksymalna liczba odcinków krzywej jeżeli nie jest ona animowana.
					\item \lstinline|int m_max_animation_segment_count| to maksymalna liczba odcinków krzywej jeżeli jest ona animowana.
					\item \lstinline|double m_segment_length| to długość pojedynczego segmentu.
					\item \lstinline|double m_segment_generation_step| to krok o jaki zwiększa się $t$ podczas generowania segmentu w metodzie \lstinline|generate_segment()|.
					\item \lstinline|bool m_is_cartesian| - prawda jeżeli krzywa jest generowana w układzie kartezjańskim, fałsz jeżeli w sferycznych.
					\item \lstinline|bool m_animate| - prawda jeżeli krzywa będzie animowana, fałsz jeżeli nie.
					\item \lstinline|Color color| - kolor obecnie generowanego odcinka, zmienia się zgodnie z algorytmem z sekcji Algorytmy.
					\item \lstinline|LissajousCurve m_curve| - krzywa na podstawie której generowane są odcinki.
					\item \lstinline|std::deque<Segment> m_queue| - kolejka do której zapisywane są kolejne segmenty.
				\end{itemize}
		\subsection{GUIMyFrame1}
			Klasa odpowiadająca za obsługę okna programu i rysowanie krzywych.
			\subsubsection{Metody}
				\begin{itemize}
					\item Metody obsługujące wydarzenia \lstinline|wxEvent| czyli przesunięcia suwaków, zaznaczone pola etc.
					\item \lstinline|void Repaint()| - metoda odpowiadająca za rysowanie krzywej.				
				\end{itemize}
			\subsubsection{Pola}
				\begin{itemize}
					\item \lstinline|int m_curve_segment_count| - liczba segmentów rysowanej krzywej.
					\item \lstinline|double m_curve_segment_length| - długość pojedynczego segmentu.
					\item \lstinline|bool m_cartesian| - czy krzywa jest rysowana we współrzędnych kartezjańskich.
					\item \lstinline|CurveGenerator m_generator| - generator krzywej.
					\item \lstinline|std::deque<Segment> m_data| - segmenty krzywej.
					\item \lstinline|Vector4 m_rotation| - aktualne kąty obrotu kamery we współrzędnych $x$, $y$, $z$. 

				\end{itemize}
\end{document}