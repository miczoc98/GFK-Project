\include{preamble}

\begin{document}
	\section{Struktury danych}
		\subsection{Deque}
			Dwukierunkowa kolejka została użyta do przechowywania elementów krzywej w postaci segmentów.
			Zaimplementowana w postaci kontenera biblioteki standardowej C++ std::deque.
		\subsection{Color}
			\begin{lstlisting}
			struct Color
			{
			    Color(int _R, int _G, int _B) : R(_R), G(_G), B(_B) {}
			    Color next();
			    Color &operator+=(int i);
			
			    int R;
			    int G;
			    int B;
			private:
			    enum cycle
			    {
			        Ru,
			        Gu,
			        Rd,
			        Bu,
			        Gd,
			        Bd,
			    };
			    cycle current_cycle = Ru;
			    int m_step = 1;
			};
			\end{lstlisting}
			Struktura przechowująca kolor w postaci kodu RGB.\\
			Metoda \lstinline|next()| pozwala wygenerować następny kolor według schematu opisanego w sekcji Algorytmy.\\
			\lstinline|operator+=(int i)| służy do zwiększenia jasności koloru poprzez dodanie \lstinline|i| do każdego ze składników RGB.
		\subsection{Point}
			\begin{lstlisting}
				struct Point
				{
				    Point() = default;
				    Point(double x, double y, double z);
				    Point as_spherical();

				    double x = 0;
				    double y = 0;
				    double z = 0;
				};
			\end{lstlisting}
			Struktura przechowuje położenie punktu w postaci trzech współrzędnych w układzie kartezjańskim lub sferycznym.\\
			Metoda \lstinline|as_spherical()| pozwala na interpretację punktu zapisanego we współrzędnych sferycznych w układzie kartezjańskim (co jest konieczne aby poprawnie wyświetlić go na ekranie).
		\subsection{Segment}
			\begin{lstlisting}
				struct Segment
				{
				    Segment(Point _begin, Point _end, Color _color) : begin(_begin), end(_end), color(_color) {}
				
				    Point begin;
				    Point end;
				    Color color;
				};
			\end{lstlisting}
			Struktura przechowująca informacje dotyczące pojedynczego odcinka w postaci: punkt początkowy, punkt końcowy, kolor odcinka.
		\subsection{Vector4}
			\begin{lstlisting}
				struct Vector4
				{
				    friend Vector4 operator*(const Vector4&, double);
				
				    Vector4();
				    Vector4 operator-(const Vector4&);
				
				    void set(double d1, double d2, double d3);
				    double get_x();
				    double get_y();
				    double get_z();
				
				    double data[4];
				};
			\end{lstlisting}
			Struktura przechowująca wektor w postaci [x, y, z, t]. Podczas inicjalizacji t jest ustawiane na 1.\\
			\lstinline|operator*| pozwala wykonać mnożenie wektora przez skalar.
		\subsection{Matrix4}
			\begin{lstlisting}
				struct Matrix4
				{
				    friend Vector4 operator*(const Matrix4, const Vector4);
				
				    Matrix4();
				    Matrix4 operator*(const Matrix4);
				
				    double data[4][4];
				};
			\end{lstlisting}
			Struktura zawierająca macierz 4x4.\\
			Pozwala na operacje mnożenia wektorowego macierz $\times$ wektor i macierz $\times$ macierz.
\end{document}